%% Generated by Sphinx.
\def\sphinxdocclass{jsbook}
\documentclass[letterpaper,10pt,dvipdfmx]{sphinxmanual}
\ifdefined\pdfpxdimen
   \let\sphinxpxdimen\pdfpxdimen\else\newdimen\sphinxpxdimen
\fi \sphinxpxdimen=.75bp\relax

\PassOptionsToPackage{warn}{textcomp}


\usepackage{cmap}
\usepackage[T1]{fontenc}
\usepackage{amsmath,amssymb,amstext}

\usepackage{times}

\usepackage{sphinx}

\fvset{fontsize=\small}
\usepackage[dvipdfm]{geometry}

% Include hyperref last.
\usepackage{hyperref}
% Fix anchor placement for figures with captions.
\usepackage{hypcap}% it must be loaded after hyperref.
% Set up styles of URL: it should be placed after hyperref.
\urlstyle{same}
\renewcommand{\contentsname}{Contents:}

\renewcommand{\figurename}{図}
\renewcommand{\tablename}{表}
\renewcommand{\literalblockname}{リスト}

\renewcommand{\literalblockcontinuedname}{前のページからの続き}
\renewcommand{\literalblockcontinuesname}{次のページに続く}
\renewcommand{\sphinxnonalphabeticalgroupname}{Non-alphabetical}
\renewcommand{\sphinxsymbolsname}{記号}
\renewcommand{\sphinxnumbersname}{Numbers}

\def\pageautorefname{ページ}

\setcounter{tocdepth}{1}



\title{ArtRepository Documentation}
\date{2018年12月22日}
\release{}
\author{Shin KURITA}
\newcommand{\sphinxlogo}{\vbox{}}
\renewcommand{\releasename}{}
\makeindex
\begin{document}

\pagestyle{empty}
\maketitle
\pagestyle{plain}
\sphinxtableofcontents
\pagestyle{normal}
\phantomsection\label{\detokenize{index::doc}}


@montblanc18による、絵画に関する忘備録。
インターネットを始め、各種媒体から収集した情報をもとに記述します。
主に西洋美術を中心に扱います。好きなのはカラヴァッジョ、次にレンブラントです。
since Dec 1 2018.


\chapter{Introduction}
\label{\detokenize{introduction:introduction}}\label{\detokenize{introduction::doc}}

\section{はじめに}
\label{\detokenize{introduction:id1}}
本資料は@montblanc18が美術作品の学習のためにまとめています。
西洋美術を中心にまとめています。


\section{著作権について}
\label{\detokenize{introduction:id2}}
美術品は著作者のもとに著作権があり、日本においては著作者の死後50年まで保証されます。
ただし、国によっては100年(どうやらメキシコの様子)のところもあるので、本資料においては、著作者の死後100年経過した作品のみ掲載します。
もしも間違い等ありましたら、ご連絡ください。


\chapter{カテゴリー}
\label{\detokenize{category:id1}}\label{\detokenize{category::doc}}

\section{古代}
\label{\detokenize{category:id2}}

\subsection{原始美術}
\label{\detokenize{category:id3}}
旧石器時代後期(約3万年前\textasciitilde{}約1万年前)以降、実生活で役立つとは考えにくい遺物・遺構が見られるようになった。
これらを総称して、原始美術と呼ぶ。
洞窟絵画などがこれに該当する。
To be written.


\subsection{メソポタミア美術}
\label{\detokenize{category:id4}}
チグリス川とユーフラテス川の間の沖積平野(現在のイラク近郊)で栄えた、
メソポタミア文明で見られた美術様式の総称である。
To be written.


\subsection{エジプト美術}
\label{\detokenize{category:id5}}
To be written.


\subsection{ギリシア美術}
\label{\detokenize{category:id6}}
To be written.


\subsection{ローマ美術}
\label{\detokenize{category:id7}}
To be written.


\section{中世}
\label{\detokenize{category:id8}}

\subsection{初期キリスト教美術}
\label{\detokenize{category:id9}}
To be written.


\subsection{ビザンティン美術}
\label{\detokenize{category:id10}}
To be written.


\subsection{初期中世美術}
\label{\detokenize{category:id11}}
To be written.


\subsection{ロマネスク美術}
\label{\detokenize{category:id12}}
To be written.


\subsection{ゴシック美術}
\label{\detokenize{category:id13}}
To be written.


\section{近世}
\label{\detokenize{category:id14}}

\subsection{イタリア初期ルネサンス美術}
\label{\detokenize{category:id15}}
ルネサンス(仏:Renaissane)は「再生」「復活」を意味する言葉であり、
古代ギリシア・ローマ時代の文化を復興しようとする文化運動である。
この活動は14世紀のイタリアから始まった。


\subsection{15世紀北方美術}
\label{\detokenize{category:id16}}
15世紀のネーデルラントにおいて、絵画に限定される美術様式。
当時ブルゴーニュ公国(現在のフランス東部からドイツ西部)に属していたネーデルランドでは、
毛織物工業と国際貿易の進行によって市民階級の台頭が起こり、豊かな経済と文化が形成された。
特に油彩技法は発色に優れ、精密で微細な質感の描写や視覚的にリアルな再現を可能とし、
これが西欧全土へと広がっていった。
一方で同時期同地域における建築や彫刻はあくまでゴシック様式の範囲内とされる。


\subsection{イタリア盛期ルネサンス美術}
\label{\detokenize{category:id17}}
イタリアルネサンスのうち、15世紀末から16世紀初頭にかけての30年間を、特に盛期ルネサンスと呼ぶ。
もともとはこの時代を古代ギリシア・ローマと並び西洋美術の完成期と見なしているからであるが、
現在ではあくまで異なる美術様式を持った時代を意味する。
中心となった都市はユリウス二世(1443-1513)のもとで活気を取り戻したローマと、東方と欧州の
貿易で富を気づいたヴェネチアである。

ルネサンス三大巨匠であるレオナルド・ダ・ヴィンチ、ミケランジェロ、ラファエロが活躍したのも、
この美術体系に含まれる。


\subsection{マニエリスム美術}
\label{\detokenize{category:id18}}
盛期ルネサンス遺構の芸術的動向を示す時代様式である。
マニエリスム自体は「様式」「手法」を意味するイタリア語に由来する。
盛期ルネサンスの特徴であった自然らしさと自然離れした要素の調和が崩れ、
自然を超えた洗練さ・芸術的技工・観念性が存在する様式である。


\subsection{北方ルネサンス美術}
\label{\detokenize{category:id19}}
To be written.


\subsection{バロック美術}
\label{\detokenize{category:id20}}
一般に17世紀の西洋美術時代様式を指す(異論はある)。
盛期ルネサンスの伝統を受け継ぎつつもより現実に即した表現が強調されるようになり、
時間の概念を取り入れた風俗画・風景画・静物画など、実社会により密着したテーマが確立する。
活動の舞台もルネサンスの中心であったヴェネチアからローマへと移り、
その後18世紀初頭にはフランスへと移っていく。


\subsubsection{バロック美術}
\label{\detokenize{Baroque:id1}}\label{\detokenize{Baroque::doc}}

\paragraph{バロック絵画}
\label{\detokenize{Baroque:id2}}
16世紀末から18世紀なかばに興ったバロック様式に分類される絵画である。
バロックは絶対王政、カトリック改革などと深い関連があり、時には一体化したものとみなされることもあるが、
バロック美術自体は絶対主義やキリスト教とは無関係に、広く親しまれている。

バロック絵画の特徴は、劇的な描写手法に豊かで深い色彩、そして強い明暗にある。
それらを存分に使って大げさで芝居がかりつつも、躍動感あふれる作品が多く制作された。
\begin{itemize}
\item {} 
オランダ
\begin{itemize}
\item {} 
フランス・ハルス(1580-1666)

\item {} 
レンブラント・ファン・レイン(1606-1669)

\item {} 
ヤン・ステーン(1626-1679)

\item {} 
ヤーコブ・ファン・ロイスダール(1628-1682)

\item {} 
ヨハネス・フェルメール(1632-1675)

\end{itemize}

\item {} 
チェコ(ボヘミア)
\begin{itemize}
\item {} 
ヴェンツェスラウス・ホラー(1607-1677)

\item {} 
カレル・スクレタ(1610-1674)

\item {} 
ペトル・ブランドル(1668-1735)

\item {} 
ヴェンツェル・ロレンツ・ライナー(1686-1743)

\end{itemize}

\item {} 
フランドル(現在のオランダ南部・ベルギー西部・フランス北部にかけての地域)
\begin{itemize}
\item {} 
ヤン・ブリューゲル(1568-1625)

\item {} 
ピーテル・パウル・ルーベンス(1577-1640)

\item {} 
フランス・スナイデルス(1579-1657)

\item {} 
ヤーコブ・ヨルダーンス(1593-1678)

\item {} 
アンソニー・ヴァン・ダイク(1599-1641)

\item {} 
ダフィット・テニールス(1610-1691)

\end{itemize}

\item {} 
フランス
\begin{itemize}
\item {} 
ジャン・ド・ボーグラン(1584-1640)

\item {} 
ジョルジュ・ド・ラ・トゥール(1593-1652)

\item {} 
ニコラ・プッサン(1594-1665)

\item {} 
アントワーヌ・ル・ナン(1599頃-1648)

\item {} 
ルイ・ル・ナン(1593頃-1648)

\item {} 
マチュー・ル・ナン(1607-1677)

\item {} 
クロード・ロラン(1600-1682)

\item {} 
イアサント・リゴー(1659-1743)

\end{itemize}

\item {} 
イタリア
\begin{itemize}
\item {} 
ルドヴィゴ・カラッチ(1555-1619)

\item {} 
アゴスティーノ・カラッチ(1557-1602)

\item {} 
アンニーバレ・カラッチ(1560-1609)

\item {} 
オラツィオ・ジェンティレスキ(1563-1639)

\item {} 
ミケランジェロ・メリージ・ダ・カラヴァッジョ(1571-1610)

\item {} 
グイド・レーニ(1575-1642)

\item {} 
グエルチーノ(1591-1666)

\item {} 
アルテミジア・ジェンティレスキ(1593-1652)

\item {} 
ピエトロ・ダ・コルトーナ(1596-1669)

\item {} 
サルヴァトル・ローザ(1615-1673)

\item {} 
アンドレア・ポッツォ(1642-1709)

\item {} 
ジョヴァンニ・バッティスタ・ティエポロ(1696-1770)

\end{itemize}

\item {} 
ポルトガル
\begin{itemize}
\item {} 
ジョセファ・ドビドス(1630-1684)

\end{itemize}

\item {} 
スペイン
\begin{itemize}
\item {} 
フランシスコ・リバルタ(1565-1628)

\item {} 
ホセ・デ・リベーラ(1591-1652)

\item {} 
フランシスコ・デ・スルバラン(1598-1664)

\item {} 
ディエゴ・ベラスケス(1599-1660)

\item {} 
アロンソ・カーノ(1601-1667)

\item {} 
バルトロメ・エステバン・ムリーリョ(1617-1682)

\item {} 
ファン・デ・バルデス・レアール(1622-1690)

\end{itemize}

\end{itemize}


\subparagraph{ミケランジェロ・メリージ・ダ・カラヴァッジョ}
\label{\detokenize{Michelangelo_Merisi_da_caravaggio_1571_1610:id1}}\label{\detokenize{Michelangelo_Merisi_da_caravaggio_1571_1610::doc}}
\noindent\sphinxincludegraphics{{Bild-Ottavio_Leoni,_Caravaggio}.jpg}

Michelangelo Merisi da Caravaggio(1571.9.28-1610.7.18)は


\paragraph{バロック彫刻}
\label{\detokenize{Baroque:id3}}
To be written.


\paragraph{バロック建築}
\label{\detokenize{Baroque:id4}}
To be written.


\subsection{ロココ美術}
\label{\detokenize{category:id21}}
1710年代から60年ごろまでのフランスの美術様式を中心とした時代様式を指す。
To be written.


\section{近代}
\label{\detokenize{category:id22}}

\subsection{新古典主義}
\label{\detokenize{category:id23}}
18世紀中頃から19世紀初頭にかけて、西欧の武術分野で支配的となった芸術思潮のこと。
より荘重な様式を求めて古典古代、特にギリシャの芸術が模範とされている。
これは装飾的かつ官能的だったバロックやロココが流行したことを反動にしている。

なお、上記の時期以外でも「新古典主義」という言葉が適用されることがある。


\subsection{ロマン主義}
\label{\detokenize{category:id24}}
新古典主義に少し遅れ、18世紀末から19世紀前半に欧州起こった精神運動である。
古典主義と対をなし、感受性や主観に重きをおいている。
恋愛賛美・民族意識の高揚・中世への憧憬という特徴は近代国家の形成を促し、その運動は芸術分野にも及んだ。
後にこのときの反動が、写実主義や自然主義をもたらした。


\subsection{写実主義}
\label{\detokenize{category:id25}}
現実をありのままに捉えようとする、美術上・文学上の主張のこと。リアリズム・レアリムス。
現実主義とも。

ルネサンス以降の美術は現実をそのまま表現することを目指してきたため、広義の写実主義と呼ぶことができる。


\subsection{印象主義}
\label{\detokenize{category:id26}}
印象派あるいは印象主義は、19世紀後半のフランスに発した絵画を中心とした芸術活動である。
初期の印象派は当時の急進派であった。
線や輪郭を描くのではなく、絵筆で自由に絵の具をのせて絵を描いた。
それまで絵は静物画や肖像画はもちろん、風景画でもアトリエで描かれていたら、印象派は時に戸外で絵を描いた。
それにより瞬間的な陽の光ではなく、それが変化していく様子を捉えるようになった。
さらに、従来のような細部に拘った描画ではなく、全体的な視覚効果を狙って混色と原色による短いストロークを並べ、鮮やかな色彩を表現した。


\subsection{象徴主義}
\label{\detokenize{category:id27}}
1870年頃のフランスとベルギーに起きた文学運動および芸術運動である。
芸術分野においては19世紀前半まで起こったロマン主義のゴシック的な側面を起源にしているが、
ロマン主義が直情的かつ反逆的であったのに対し、象徴主義は静的かつ儀式的なおものであった


\subsection{ポスト印象派}
\label{\detokenize{category:id28}}
印象派のあとに、フランスを中心として主に1880年代から活躍した画家たちを指す言葉である。
『印象派の後に続く画家たち』を指す言葉であり、様式的な統一性はない。
もちろん、『印象派後期の画家たち』ではない。
一般的には、ゴッホやゴーギャン、セザンヌを指す。


\section{現代}
\label{\detokenize{category:id29}}

\subsection{ベル・エポック}
\label{\detokenize{category:id30}}
1900年から第一次世界大戦までの華やかな時代を指すフランス語(ベル・エポック=良き時代)で、反映と平和を享受した時代である。
アール・ヌーヴォー、アール・デコ、ユーゲント・シュティールなど多彩な幻術活動がヨーロッパを席巻した。


\subsubsection{現代建築}
\label{\detokenize{category:id31}}
To be written.


\subsubsection{現代彫刻}
\label{\detokenize{category:id32}}
To be written.


\subsubsection{現代絵画}
\label{\detokenize{category:id33}}
1905年、サロン・ドートンヌ(フランスの展示会)に鮮烈な色使いの作品を発表した画家たちは、フォーヴィスムという運動を起こした。
強烈な色使いゆえ当初は酷評されたフォーヴィスムであるが、多くの画家が伝統に縛られない色彩に共鳴していった。
しかし結局は大きな流行にならず、先のサロン・ドートンヌで話題となった1905年をピークに、この運動は減衰していった。

フランスでフォーヴィスムが誕生したのと同年、ドイツで前衛絵画グループが結成された。
このグループは後にドイツ表現主義の先駆けとされ、強烈な色彩表現を行う点でフォーヴィスムと共通する。

その後ピカソやブラックによってキュビスムが興される。
1907年にピカソが発表した『アビニヨンの娘たち』は非常に有名である。

第一次世界大戦後にはシュルレアリスムが登場し、サルバトーレ・ダリらが活躍する。
彼らは新たな絵画技法を次々と生み出しながら、夢と現実が混ざった世界観を描いていった。

その後も様々な価値観や美術様式が生まれ、現在に至るまで非常に多様化している。


\chapter{Indices and tables}
\label{\detokenize{index:indices-and-tables}}\begin{itemize}
\item {} 
\DUrole{xref,std,std-ref}{genindex}

\item {} 
\DUrole{xref,std,std-ref}{modindex}

\item {} 
\DUrole{xref,std,std-ref}{search}

\end{itemize}



\renewcommand{\indexname}{索引}
\printindex
\end{document}